\let\negmedspace\undefined
\let\negthickspace\undefined
%\RequirePackage{amsmath}
\documentclass[journal,12pt,twocolumn]{IEEEtran}
%
% \usepackage{setspace}
 \usepackage{gensymb}
%\doublespacing
 \usepackage{polynom}
%\singlespacing
%\usepackage{silence}
%Disable all warnings issued by latex starting with "You have..."
%\usepackage{graphicx}
\usepackage{amssymb}
%\usepackage{relsize}
\usepackage[cmex10]{amsmath}
%\usepackage{amsthm}
%\interdisplaylinepenalty=2500
%\savesymbol{iint}
%\usepackage{txfonts}
%\restoresymbol{TXF}{iint}
%\usepackage{wasysym}
\usepackage{amsthm}
%\usepackage{pifont}
%\usepackage{iithtlc}
% \usepackage{mathrsfs}
% \usepackage{txfonts}
 \usepackage{stfloats}
% \usepackage{steinmetz}
 \usepackage{bm}
% \usepackage{cite}
% \usepackage{cases}
% \usepackage{subfig}
%\usepackage{xtab}
\usepackage{longtable}
%\usepackage{multirow}
%\usepackage{algorithm}
%\usepackage{algpseudocode}
\usepackage{enumitem}
 \usepackage{mathtools}
 \usepackage{tikz}
% \usepackage{circuitikz}
% \usepackage{verbatim}
%\usepackage{tfrupee}
\usepackage[breaklinks=true]{hyperref}
%\usepackage{stmaryrd}
%\usepackage{tkz-euclide} % loads  TikZ and tkz-base
%\usetkzobj{all}
\usepackage{listings}
    \usepackage{color}                                            %%
    \usepackage{array}                                            %%
    \usepackage{longtable}                                        %%
    \usepackage{calc}                                             %%
    \usepackage{multirow}                                         %%
    \usepackage{hhline}                                           %%
    \usepackage{ifthen}                                           %%
  %optionally (for landscape tables embedded in another document): %%
    \usepackage{lscape}     
% \usepackage{multicol}
% \usepackage{chngcntr}
%\usepackage{enumerate}
\usepackage{tfrupee}

%\usepackage{wasysym}
%\newcounter{MYtempeqncnt}
\DeclareMathOperator*{\Res}{Res}
\DeclareMathOperator*{\equals}{=}
%\renewcommand{\baselinestretch}{2}
\renewcommand\thesection{\arabic{section}}
\renewcommand\thesubsection{\thesection.\arabic{subsection}}
\renewcommand\thesubsubsection{\thesubsection.\arabic{subsubsection}}

\renewcommand\thesectiondis{\arabic{section}}
\renewcommand\thesubsectiondis{\thesectiondis.\arabic{subsection}}
\renewcommand\thesubsubsectiondis{\thesubsectiondis.\arabic{subsubsection}}

% correct bad hyphenation here
\hyphenation{op-tical net-works semi-conduc-tor}
\def\inputGnumericTable{}                                 %%

\lstset{
%language=C,
frame=single, 
breaklines=true,
columns=fullflexible
}
%\lstset{
%language=tex,
%frame=single, 
%breaklines=true
%}
\begin{document}

%


\newtheorem{theorem}{Theorem}[section]
\newtheorem{problem}{Problem}
\newtheorem{proposition}{Proposition}[section]
\newtheorem{lemma}{Lemma}[section]
\newtheorem{corollary}[theorem]{Corollary}
\newtheorem{example}{Example}[section]
\newtheorem{definition}[problem]{Definition}
%\newtheorem{thm}{Theorem}[section] 
%\newtheorem{defn}[thm]{Definition}
%\newtheorem{algorithm}{Algorithm}[section]
%\newtheorem{cor}{Corollary}
\newcommand{\BEQA}{\begin{eqnarray}}
\newcommand{\EEQA}{\end{eqnarray}}
\newcommand{\define}{\stackrel{\triangle}{=}}
\newcommand*\circled[1]{\tikz[baseline=(char.base)]{
    \node[shape=circle,draw,inner sep=2pt] (char) {#1};}}
\bibliographystyle{IEEEtran}
%\bibliographystyle{ieeetr}
\providecommand{\mbf}{\mathbf}
\providecommand{\pr}[1]{\ensuremath{\Pr\left(#1\right)}}
\providecommand{\qfunc}[1]{\ensuremath{Q\left(#1\right)}}
\providecommand{\sbrak}[1]{\ensuremath{{}\left[#1\right]}}
\providecommand{\lsbrak}[1]{\ensuremath{{}\left[#1\right.}}
\providecommand{\rsbrak}[1]{\ensuremath{{}\left.#1\right]}}
\providecommand{\brak}[1]{\ensuremath{\left(#1\right)}}
\providecommand{\lbrak}[1]{\ensuremath{\left(#1\right.}}
\providecommand{\rbrak}[1]{\ensuremath{\left.#1\right)}}
\providecommand{\cbrak}[1]{\ensuremath{\left\{#1\right\}}}
\providecommand{\lcbrak}[1]{\ensuremath{\left\{#1\right.}}
\providecommand{\rcbrak}[1]{\ensuremath{\left.#1\right\}}}
\theoremstyle{remark}
\newtheorem{rem}{Remark}
\newcommand{\sgn}{\mathop{\mathrm{sgn}}}
\providecommand{\fourier}{\overset{\mathcal{F}}{ \rightleftharpoons}}
%\providecommand{\hilbert}{\overset{\mathcal{H}}{ \rightleftharpoons}}
\providecommand{\system}{\overset{\mathcal{H}}{ \longleftrightarrow}}
	%\newcommand{\solution}[2]{\textbf{Solution:}{#1}}
\newcommand{\solution}{\noindent \textbf{Solution: }}
\newcommand{\cosec}{\,\text{cosec}\,}
\providecommand{\dec}[2]{\ensuremath{\overset{#1}{\underset{#2}{\gtrless}}}}
\newcommand{\myvec}[1]{\ensuremath{\begin{pmatrix}#1\end{pmatrix}}}
\newcommand{\mydet}[1]{\ensuremath{\begin{vmatrix}#1\end{vmatrix}}}
%\numberwithin{equation}{section}
%\numberwithin{figure}{section}
%\numberwithin{table}{section}
%\numberwithin{equation}{subsection}
%\numberwithin{problem}{section}
%\numberwithin{definition}{section}
\makeatletter
\@addtoreset{figure}{problem}
\makeatother
\let\StandardTheFigure\thefigure
\let\vec\mathbf
%\renewcommand{\thefigure}{\theproblem.\arabic{figure}}
%\renewcommand{\thefigure}{\theproblem}
%\setlist[enumerate,1]{before=\renewcommand\theequation{\theenumi.\arabic{equation}}
%\counterwithin{equation}{enumi}
%\renewcommand{\theequation}{\arabic{subsection}.\arabic{equation}}
\def\putbox#1#2#3{\makebox[0in][l]{\makebox[#1][l]{}\raisebox{\baselineskip}[0in][0in]{\raisebox{#2}[0in][0in]{#3}}}}
     \def\rightbox#1{\makebox[0in][r]{#1}}
     \def\centbox#1{\makebox[0in]{#1}}
     \def\topbox#1{\raisebox{-\baselineskip}[0in][0in]{#1}}
     \def\midbox#1{\raisebox{-0.5\baselineskip}[0in][0in]{#1}}
\title{
	%\logo{
%Computational Approach to School Geometry
	Assignment
%	}
}
\author{ Nishanth Bhoomi CS21BTECH11040% <-this % stops a space 
                                                                                                                                                                                                                                                                                                        }
\graphicspath{{figures/}}
%\title{
%	\logo{Matrix Analysis through Octave}{\begin{center}\includegraphics[scale=.24]{tlc}\end{center}}{}{HAMDSP}
%}
% paper title
% can use linebreaks \\ within to get better formatting as desired
%\title{Matrix Analysis through Octave}
%
%
% author names and IEEE memberships
% note positions of commas and nonbreaking spaces ( ~ ) LaTeX will not break
% a structure at a ~ so this keeps an author's name from being broken across
% two lines.
% use \thanks{} to gain access to the first footnote area
% a separate \thanks must be used for each paragraph as LaTeX2e's \thanks
% was not built to handle multiple paragraphs
%
%\author{<-this % stops a space
%\thanks{}}
%}
% note the % following the last \IEEEmembership and also \thanks - 
% these prevent an unwanted space from occurring between the last author name
% and the end of the author line. i.e., if you had this:
% 
% \author{....lastname \thanks{...} \thanks{...} }
%                     ^------------^------------^----Do not want these spaces!
%
% a space would be appended to the last name and could cause every name on that
% line to be shifted left slightly. This is one of those "LaTeX things". For
% instance, "\textbf{A} \textbf{B}" will typeset as "A B" not "AB". To get
% "AB" then you have to do: "\textbf{A}\textbf{B}"
% \thanks is no different in this regard, so shield the last } of each \thanks
% that ends a line with a % and do not let a space in before the next \thanks.
% Spaces after \IEEEmembership other than the last one are OK (and needed) as
% you are supposed to have spaces between the names. For what it is worth,
% this is a minor point as most people would not even notice if the said evil
% space somehow managed to creep in.
%\WarningFilter{latex}{LaTeX Warning: You have requested, on input line 117, version}
% The paper headers
%\markboth{Journal of \LaTeX\ Class Files,~Vol.~6, No.~1, January~2007}%
%{Shell \MakeLowercase{\textit{et al.}}: Bare Demo of IEEEtran.cls for Journals}
% The only time the second header will appear is for the odd numbered pages
% after the title page when using the twoside option.
% 
% * Note that you probably will NOT want to include the author's *
% * name in the headers of peer review papers.                   *
% You can use \ifCLASSOPTIONpeerreview for conditional compilation here if
% you desire.
% If you want to put a publisher's ID mark on the page you can do it like
% this:
%\IEEEpubid{0000--0000/00\$00.00~\copyright~2007 IEEE}
% Remember, if you use this you must call \IEEEpubidadjcol in the second
% column for its text to clear the IEEEpubid mark.
% make the title area
\maketitle
%\renewcommand{\theequation}{\theenumi}
%\begin{abstract}
%%\boldmath
%In this letter, an algorithm for evaluating the exact analytical bit error rate  (BER)  for the piecewise linear (PL) combiner for  multiple relays is presented. Previous results were available only for upto three relays. The algorithm is unique in the sense that  the actual mathematical expressions, that are prohibitively large, need not be explicitly obtained. The diversity gain due to multiple relays is shown through plots of the analytical BER, well supported by simulations. 
%
%\end{abstract}
% IEEEtran.cls defaults to using nonbold math in the Abstract.
% This preserves the distinction between vectors and scalars. However,
% if the journal you are submitting to favors bold math in the abstract,
% then you can use LaTeX's standard command \boldmath at the very start
% of the abstract to achieve this. Many IEEE journals frown on math
% in the abstract anyway.
% Note that keywords are not normally used for peerreview papers.
%\begin{IEEEkeywords}
%Cooperative diversity, decode and forward, piecewise linear
%\end{IEEEkeywords}
% For peer review papers, you can put extra information on the cover
% page as needed:
% \ifCLASSOPTIONpeerreview
% \begin{center} \bfseries EDICS Category: 3-BBND \end{center}
% \fi
%
% For peerreview papers, this IEEEtran command inserts a page break and
% creates the second title. It will be ignored for other modes.
%\IEEEpeerreviewmaketitle
%\begin{abstract}
%This document contains my solution for Assignment 2 
%\end{abstract}




%template ends here





%main text begins

\begin{flushleft}
\section{Uniform Random Numbers}
\begin{enumerate}[label=\thesection.\arabic*
,ref=\thesection.\theenumi]
\item Generate $10^6$ samples of $U$ using a C program and save into a file called uni.dat .
\\
\solution Download the following files and execute the  C program.
\begin{lstlisting}
wget https://github.com/NishanthBhoomi/Random__Numbers/blob/main/1.1/exrand.c
wget https://github.com/NishanthBhoomi/Random__Numbers/blob/main/1.1/coeffs.h
\end{lstlisting}

We can obtain the $10^6$ samples using 
\begin{lstlisting}
gcc exrand.c -lm
./a.out
\end{lstlisting}
\item
Load the uni.dat file into python and plot the empirical CDF of $U$ using the samples in uni.dat. The CDF is defined as
\begin{align}
F_{U}(x) = \pr{U \le x}
\end{align}
\\
\solution  The following code plots Fig. \ref{fig:uni_cdf}
\begin{lstlisting}
wget https://github.com/NishanthBhoomi/Random__Numbers/blob/main/1.2/uni_cdf.py
\end{lstlisting}
By running the python code
\begin{lstlisting}
python3 uni_cdf.py
\end{lstlisting}
\begin{figure}
\centering
\includegraphics[width=\columnwidth]{./figs/uni_cdf.png}
\caption{The CDF of $U$}
\label{fig:uni_cdf}
\end{figure}

\item
Find a  theoretical expression for $F_{U}(x)$.

\solution
Let $X$ be a uniform random variable in the interval $[a,b]$.
\begin{align}
\pr{X \in [x_{1},x_{2}]} = \frac{x_{2}-x_{1}}{b-a} 
\end{align}
\begin{flushright}
where $a \leq x_{1} \leq x_{2} \leq b$
\end{flushright}

\begin{align*}
 F_{X}(x)= \pr{X<x} \\
 F_{X}(x)= 0 \ \text{    for} \ x<a  \\
 F_{X}(x)= 1 \ \text{    for} \ x>b  \\
\end{align*}

For $a \leq x \leq b$ we have
\begin{align*}
\begin{split}
F_{X}(x) & = \pr{X<x} \\
 & = \pr{X<a} + \pr{a<X<x} \\
 & = 0 + \pr{a<X<x} \\
 & = \pr{X \in [a,x]} \\
 & = \frac{x-a}{b-a}
\end{split}
\end{align*}

\begin{equation}
F_{X}(x)=
\begin{cases}
0 &\text{if } x < a \\
\frac{x-a}{b-a} &\text{if } a \leq x \leq b \\
1 &\text{if } x > b \\
\end{cases}
\end{equation}

\begin{equation}
F_{U}(x)=
\begin{cases}
0 &\text{if } x < 0 \\
x &\text{if } 0 \leq x \leq 1 \\
1 &\text{if } x > 1 \\
\end{cases}
\end{equation}

\item
The mean of $U$ is defined as
%
\begin{equation}
E\sbrak{U} = \frac{1}{N}\sum_{i=1}^{N}U_i
\end{equation}
%
and its variance as
%
\begin{equation}
\text{var}\sbrak{U} = E\sbrak{U- E\sbrak{U}}^2 
\end{equation}
Write a C program to  find the mean and variance of $U$. \\
\solution
\begin{lstlisting}
wget https://github.com/NishanthBhoomi/Random__Numbers/blob/main/1.4/mean_var.c
\end{lstlisting}

Use below command to run file,
\begin{lstlisting}
gcc mean_var.c -lm
./a.out
\end{lstlisting}

\item Verify your result theoretically given that

%
\begin{equation}
E\sbrak{U^k} = \int_{-\infty}^{\infty}x^kdF_{U}(x)
\end{equation}

\solution
\begin{equation}
\begin{split}
E\sbrak{U} & = \int_{-\infty}^{\infty}xdF_{U}(x) \\
           & = \int_{0}^{1}xdx \\
           & = \frac{1}{2}
\end{split}https://github.com/NishanthBhoomi/Random__Numbers/blob/main/1.4/mean_var.c
\end{equation}

\begin{equation} \label{eq:9}
\begin{split}
\text{var}\sbrak{U} & = E\sbrak{U - E\sbrak{U}}^2 \\
					& = E\sbrak{U^2} - 2UE\sbrak{U} + \brak{E\sbrak{U}}^2 \\
					& = E\sbrak{U^{2}} - 2E\sbrak{U}E\sbrak{U} + \brak{E\sbrak{U}}^2 \\
					& = E\sbrak{U^{2}} - \brak{E\sbrak{U}}^2
\end{split}
\end{equation}

\begin{equation}
\begin{split}
E\sbrak{U^2} & = \int_{-\infty}^{\infty}x^2dF_{U}(x) \\
			 & = \int_{0}^{1}x^2dx \\
			 & = \frac{1}{3}
\end{split}
\end{equation}https://github.com/NishanthBhoomi/Random__Numbers/blob/main/1.2/uni_cdf.py

From \eqref{eq:9}
\begin{equation}
\begin{split}
\text{var}\sbrak{U} & = E\sbrak{U^{2}} - \brak{E\sbrak{U}}^2 \\
					& = \frac{1}{3} - {\frac{1}{2}}^2 \\
					& = \frac{1}{12}
\end{split}
\end{equation}

\end{enumerate}

%\begin{figure}[h]
%\centering
%\includegraphics[width=\columnwidth]{Figures/graph1.png}
%\caption{Sketch of $ y = |x+4|$}
%\label{graph}
%\end{figure}



%\begin{figure}[h]
%\centering
%\includegraphics[width=\columnwidth]{Figures/graph2.png}
%\caption{Sketch of $ y = |x+4|$ with P.O.I with $x=-6$ and $x=0$ } 
%\label{graph}
%\end{figure}

\end{flushleft}
\section{Central Limit Theorem}
%
\begin{enumerate}[label=\thesection.\arabic*
,ref=\thesection.\theenumi]
%
\item
Generate $10^6$ samples of the random variable
%
\begin{equation}
X = \sum_{i=1}^{12}U_i -6
\end{equation}
%
using a C program, where $U_i, i = 1,2,\dots, 12$ are  a set of independent uniform random variables between 0 and 1
and save in a file called gau.dat
\\
\solution
\begin{lstlisting}
wget https://github.com/NishanthBhoomi/Random__Numbers/blob/main/1.1/exrand.c
wget https://github.com/NishanthBhoomi/Random__Numbers/blob/main/1.1/coeffs.h

\end{lstlisting}
Running the above codes generates uni.dat and gau.dat file.
Use the command 
\begin{lstlisting}
gcc exrand.c -lm
.\a.out
\end{lstlisting}
%
\item
Load gau.dat in python and plot the empirical CDF of $X$ using the samples in gau.dat. What properties does a CDF have?
\\
\solution 
The CDF of $X$ is plotted in \ref{fig:gau_cdf},Properties of the CDF:
\begin{itemize}
\item $\Phi(x)=P(Z \leq x)= \frac{1}{\sqrt{2 \pi}} \int_{-\infty}^{x}\exp\left\{-\frac{u^2}{2}\right\} du$
\item $\lim \limits_{x\rightarrow \infty} \Phi(x)=1, \hspace{5pt} \lim \limits_{x\rightarrow -\infty} \Phi(x)=0$
\item  $\Phi(0)=\frac{1}{2}$
\item  $\Phi(-x)=1-\Phi(x)$
\end{itemize}
\begin{figure}[h]
\centering
\includegraphics[width=\columnwidth]{./gau_cdf}
\caption{The CDF of $X$}
\label{fig:gau_cdf}
\end{figure}
\item
Load gau.dat in python and plot the empirical PDF of $X$ using the samples in gau.dat. The PDF of $X$ is defined as
\begin{align}
p_{X}(x) = \frac{d}{dx}F_{X}(x)
\end{align}
What properties does the PDF have?
\\
\begin{figure}[h]
\centering
\includegraphics[width=\columnwidth]{./gauss_pdf}
\caption{The PDF of $X$}
\label{fig:gauss_pdf}
\end{figure}
\\
\solution The PDF of $X$ is plotted in \ref{fig:gauss_pdf} using the code below
\begin{lstlisting}
https://github.com/NishanthBhoomi/Random__Numbers/blob/main/2.3/gauss_pdf.py
\end{lstlisting}
Use the below command to run the code:
\begin{lstlisting}
python3 gauss_pdf.py
\end{lstlisting}
Properties of PDF:
\begin{itemize}
\item PDF is symmetric about $x=0$\
\item graph is bell shaped\
\item mean of graph is situated at the apex point of the bell\
\end{itemize}
\item Find the mean and variance of $X$ by writing a C program.\\
\solution
Running the below code gives Mean = -0.000417 Variance= 0.999902
 \begin{lstlisting}
https://github.com/NishanthBhoomi/Random__Numbers/blob/main/2.4/mean_var(gau).c
\end{lstlisting}
Command used:
\begin{lstlisting}
gcc mean_var(gau).c -lm
./a.out
\end{lstlisting}
\item Given that 
\begin{align}
p_{X}(x) = \frac{1}{\sqrt{2\pi}}\exp\brak{-\frac{x^2}{2}}, -\infty < x < \infty,
\end{align}
repeat the above exercise theoretically.
%
\end{enumerate}
Given ,$p_{X}(x)=\frac{1}{\sqrt{2\pi}} e^{\frac{-x^2}{2}}$\
\begin{align}
 &E[x]=\int_{-\infty}^{\infty} x p_{X}(x) dx\\
 &=\int_{-\infty}^{\infty} \frac{1}{\sqrt{2 \pi}} x e^{-\frac{-x^2}{2}}\\
  &\because x e^{-\frac{-x^2}{2}} \text{is a odd function},\\
  \nonumber
   &E[x]=0\\
 &E[x^2]=\int_{-\infty}^{\infty} x^2 p_{X}(x) dx\\
 &=\int_{-\infty}^{\infty} \frac{1}{\sqrt{2\pi}} x(xe^{-\frac{-x^2}{2}}) dx
\end{align}
  Using integration by parts:
  \begin{align}
   \label{eq:eq1}
 & =x\int xe^{-\frac{-x^2}{2}} dx-\int\frac{d(x)}{dx} \int xe^{-\frac{-x^2}{2}}dx\\
 &I=\int x e^{-\frac{-x^2}{2}}\\
 &\text{Let} \frac{x^2}{2}=t \\
 &\implies x dx=dt\\
 &\implies =\int e^{-t} dt=-e^{-t} +c\\
 \label{eq:eq2}
 &\therefore \int x e^{-\frac{-x^2}{2}}=-e^{-\frac{-x^2}{2}} +c
 \end{align}
 Using \eqref{eq:eq2} in \eqref{eq:eq1}\
 \begin{align}
&= -x e^{-\frac{-x^2}{2}}+\int e^{-\frac{-x^2}{2}} dx\\
&\text{Also} ,\int_{-\infty}^{\infty} e^{-\frac{-x^2}{2}} dx=\sqrt{2 \pi} \\
&\therefore \text{substituting limits we get}, E[x^2]=1\\
 &Var(X)=E[x^2]-(E[x])^2=1-0
 \end{align}
\section{From Uniform to Other}
\begin{enumerate}[label=\thesection.\arabic*
,ref=\thesection.\theenumi]
%
\item
Generate samples of 
%
\begin{equation}
V = -2\ln\brak{1-U}
\end{equation}
%
and plot its CDF.\\ 
 \begin{figure}[h]
\includegraphics[width=0.5\textwidth]{V_cdf.png}
\caption{CDF for (3)}
\label{fig:V}
\end{figure}
\\ 
\solution
Running the below code generates samples of V from file uni.dat(U).
\begin{lstlisting}
https://github.com/NishanthBhoomi/Random__Numbers/blob/main/3.1/V.py
\end{lstlisting}
Use the below command in the terminal to run the code:
\begin{lstlisting}
python3 V.py
\end{lstlisting}
 
Now these samples are used to plot \eqref{fig:V} by running the below code,
\begin{lstlisting}
https://github.com/NishanthBhoomi/Random__Numbers/blob/main/3.1/V_cdf.py
\end{lstlisting}
Use the below command to run the code:
\begin{lstlisting}
python3 V_cdf.py
\end{lstlisting}
\item Find a theoretical expression for $F_V(x)$.
\begin{align}
 &F_{V}(x)=P(V \leq x)\\
 &=P(-2 ln(1-U) \leq x)\\
 &=P(1-e^{\frac{-x}{2}} \geq U)\\
 &P(U<x)=\int_{0}^{x} dx=x\\
 &\therefore P(1-e^{\frac{-x}{2}} \geq U)=1-e^{\frac{-x}{2}}, \forall x\geq 0 \\ 
 \nonumber
 \end{align}
 %
%\item
%Generate the Rayleigh distribution from Uniform. Verify your result through graphical plots.
\end{enumerate}
\section{Triangular Distribution}
\begin{enumerate}[label=\thesection.\arabic*
,ref=\thesection.\theenumi]
%
\item Generate 
	\begin{align}
		T = U_1+U_2
	\end{align}
\solution 
Run the below code to generate T.dat
\begin{lstlisting}
wget https://github.com/NishanthBhoomi/Random__Numbers/blob/main/4.1/T_gen_dat.c
\end{lstlisting}
Run the command below in the terminal 
\begin{lstlisting}
gcc T_gen_dat.c -lm
./a.out
\end{lstlisting}
\item Find the CDF of $T$.
\begin{align}
F_{T}(t)=P(T<t)
\\=P(U_1 +U_2 <t)
\end{align}
we know that $0\leq U_1 \leq 1$ and $0\leq U_2 \leq 1$\\
$\therefore 0\leq U_1 + U_2 \leq 2$, so\\
 $\forall t>2, P(U_1 +U_2 <t)=1$\\
 $\forall t<0, P(U_1 +U_2 <t)=0$\\
   for $0\leq t \leq 2$ let us split it into 2 cases, for $0 \leq t\leq 1$ and $1 <t \leq2$\\
    \begin{figure}[h]
\includegraphics[width=0.5\textwidth]{T_plot}
\caption{Plot}
\label{fig:T_plot}
\end{figure}
\\
The above figure is produced by the following code
\begin{lstlisting}
wget https://github.com/NishanthBhoomi/Random__Numbers/blob/main/4.2/T_plot.py
\end{lstlisting}
Run the following command in the terminal to run the code
\begin{lstlisting}
python3 T_plot.py
\end{lstlisting}
From Fig \eqref{fig:T_plot}
\begin{align}
&P(U_1+U_2<t, 0\leq t \leq 1)=\frac{ \Delta(EOF)}{\Delta(AEOD)}\\
&=\frac{t^2}{2}\\
&P(U_1+U_2<t, 1\leq t \leq 2)=\frac{\Delta(ABC)}{\Delta(AEOD)}\\
&=1-\frac{(2-t)^{2}}{2}\\
&\therefore F_{T}(t)=P(U_1 +U_2<t)=
\begin{cases}
0 & t<0\\
\frac{t^2}{2} & 0\leq t \leq 1\\
1-\frac{(2-t)^{2}}{2} & 1< t \leq 2\\
1 & t>2
\end{cases}
\end{align}
\item Find the PDF of $T$.\\
\solution 
\begin{align}
P_{T}(t)=\frac{d(F_{T}(t))}{dt}\\
\therefore P_{T}(t)=
\begin{cases}
0 & t<0\\
t & 0\leq t \leq 1\\
2-t  & 0< t \leq 2\\
0 & t>2 
\end{cases}    
\end{align}
\item Find the theoretical expressions for the PDF and CDF of $T$.
\\
\solution
\begin{align}
P_{T}(t)=
\begin{cases}
0 & t<0\\
t & 0\leq t \leq 1\\
2-t  & 0< t \leq 2\\
0 & t>2 
\end{cases} 
\\   
F_{T}(t)=
\begin{cases}
0 & t<0\\
\frac{t^2}{2} & 0\leq t \leq 1\\
1-\frac{(2-t)^{2}}{2} & 1< t \leq 2\\
1 & t>2
\end{cases}
\end{align}
\item Verify your results through a plot. 
\\
\solution 
 \begin{figure}[h]
\includegraphics[width=0.5\textwidth]{T_cdf}
\caption{CDF for (4)}
\label{fig:T_CDF}
\end{figure}
Run the below code to get the cdf
\begin{lstlisting}
wget https://github.com/NishanthBhoomi/Random__Numbers/blob/main/4.5/T_cdf.py
\end{lstlisting}
Use the following command in the terminal to run the code
\begin{lstlisting}
python3 T_cdf.py
\end{lstlisting}
\begin{figure}[h]
\includegraphics[width=0.5\textwidth]{T_pdf}
\caption{PDF for (4)}
\label{fig:T_PDF}
\end{figure}
Run the below code to get the pdf
\begin{lstlisting}
wget https://github.com/NishanthBhoomi/Random__Numbers/blob/main/4.5/T_pdf.py
\end{lstlisting}
Use the following command in the terminal to run the code
\begin{lstlisting}
python3 T_pdf.py
\end{lstlisting}
\end{enumerate}
\section{Maximul Likelihood}
\begin{enumerate}[label=\thesection.\arabic*
,ref=\thesection.\theenumi]
\item Generate equiprobable $X \in \cbrak{1,-1}$.\\
\solution
Run the below code,
\begin{lstlisting}
wget https://github.com/NishanthBhoomi/Random__Numbers/blob/main/5.1/bernoulli.c
\end{lstlisting}
Use the below command in the terminal to run the code
\begin{lstlisting}
gcc bernoulli.c -lm
./a.out
\end{lstlisting}
\item Generate 
\begin{equation}
Y = AX+N,
\end{equation}
		where $A = 5$ dB,  and $N \sim \gauss{0}{1}$.\\
\solution
Run the below code for generating samples of Y,
\begin{lstlisting}
wget https://github.com/NishanthBhoomi/Random__Numbers/blob/main/5.2/Ygen.c
\end{lstlisting}
Use the below command in the terminal to run the code
\begin{lstlisting}
gcc Ygen.c -lm
./a.out
\end{lstlisting}
\item Plot $Y$ using a scatter plot.\\
	\solution
	\begin{figure}[h]
\includegraphics[width=0.5\textwidth]{Yplot}
\caption{plot for (5.3)}
\label{fig:Y_Plot}
\end{figure}
\\
	Run the following code to generate the scatter plot
	\begin{lstlisting}
wget https://github.com/NishanthBhoomi/Random__Numbers/blob/main/5.3/Yplot.py
	\end{lstlisting}
	Use the below command to run the code,
	\begin{lstlisting}
     python3 Yplot.py 
	\end{lstlisting}
\item Guess how to estimate $X$ from $Y$.\\
	\solution
	if the received signal is greater than 0, then the receiver assumes $s_1$ was transmitted.\\
if the received signal is less than or equal to 0, then the receiver assumes $s_0$ was transmitted, where $s_0$ and $s_1$ are cases of $X=1$and $X=-1$ respectively where threshold 0 is taken to be the decision boundary.
\begin{align}
y>0 \implies s_1\\
y\leq 0 \implies s_0
\end{align}
\item
\label{ml-ch4_sim}
Find 
\begin{equation}
	P_{e|0} = \pr{\hat{X} = -1|X=1}
\end{equation}
and 
\begin{equation}
	P_{e|1} = \pr{\hat{X} = 1|X=-1}
\end{equation}
\solution 
Here $s_1$ and $s_2$ are equally probable ie, $p(s_1)=p(s_0)=\frac{1}{2}$ \\
\begin{align}
&Q(x)=\frac{1}{\sqrt{2\pi}} \int_{x}^{\infty} e^{\frac{-x^{2}}{2} } dx\\
&p(e|s_1)=\frac{1}{\sqrt{2 \pi}} \int_{-\infty}^{0} e^{-\frac{(y-A)^2}{2}} dy
\nonumber \\
&=Q(A)\\
&p(e|s_0)=\frac{1}{\sqrt{2 \pi}} \int_{0}^{\infty} e^{-\frac{(y+A)^{2}}{2}} dy
\nonumber \\
&=Q(A)
\end{align}
%
\item Find $P_e$ assuming that $X$ has equiprobable symbols.\\
\solution
Total probability of bit error:
\begin{align}
&P_{e}=p(s_1)p(e|s_1)+p(s_0)p(e|s_0)\\
&=\frac{1}{2}[Q(A)+Q(A)]\\
&\because p(s_1)=p(s_0)=\frac{1}{2},\text{X has equiprobable symbols}\\
\nonumber
&=Q(A)
&=Q(5) \because A=5
\end{align}
%
\item
Verify by plotting  the theoretical $P_e$ with respect to $A$ from 0 to 10 dB. \\
\solution
\begin{figure}[!ht]
\includegraphics[width=0.5\textwidth]{Pplot}
\caption{plot for (5.7)}
\label{fig:Plt}
\end{figure}
	\begin{lstlisting}
wget https://github.com/NishanthBhoomi/Random__Numbers/blob/main/5.7/Pplot.py
	\end{lstlisting}
	Use the below command to run the code,
	\begin{lstlisting}
     python3 Pplot.py 
	\end{lstlisting}
\item Now, consider a threshold $\delta$  while estimating $X$ from $Y$. Find the value of $\delta$ that minimizes the theoretical $P_e$.\\
\solution
Threshold=$\delta$, \\
\begin{align}
 &y>\delta \implies s_1\\
 &y \leq \delta \implies s_0\\
 &p(e|s_1)=\frac{1}{\sqrt{2 \pi}} \int_{-\infty}^{\delta} e^{-\frac{(y-A)^{2}}{2}} dy\\
 \nonumber
 &p(e|s_0)=\frac{1}{\sqrt{2 \pi}} \int_{\delta}^{\infty} e^{-\frac{(y+A)^{2}}{2}} dy\\
\nonumber
&P_e=\frac{1}{2\sqrt{2 \pi}}{( \int_{-\infty}^{\delta} e^{-\frac{(y-A)^{2}}{2}}dy+ \int_{\delta}^{\infty} e^{-\frac{(y+A)^{2}}{2}} dy)}\\
&P_e=\frac{Q(\delta+A)+Q(A-\delta)}{2}\\
&P_e=f(\delta)\\
&\text{to minimize} P_e, \frac{d(f(\delta))}{d\delta}=0 ~\text{and} f"(\delta)>0\\
&e^{\frac{-(A-\delta)^{2}}{2}}-e^{\frac{-(A+\delta)^{2}}{2}}=0\\
&\therefore A-\delta=A+\delta, \implies \delta=0\\
&f"(\delta)=k((A-\delta)e^{\frac{-(A-\delta)^{2}}{2}}+(A+\delta)e^{\frac{-(A+\delta)^{2}}{2}})>0
\end{align}
\item Repeat the above exercise when 
	\begin{align}
		p_{X}(0) = p
	\end{align}
\solution
 $p_{X}(0)=p$ \\
 $\implies p_{X}(1)=1-p$
 \begin{align}
 P_{e}=p P(e|s_0)+(1-p)P(e|s_1)\\
 =p Q(A+\delta)+(1-p)Q(A-\delta)\\
 \frac{d(P_{e})}{d(\delta)}=0\\
 \implies e^\frac{(A+\delta)^{2}-(A-\delta)^{2}}{2}=\frac{p}{1-p}\\
 \therefore \delta=\frac{1}{2A}log(\frac{p}{1-p})\\
  \frac{d(P_{e})}{d(\delta)}\quad at \quad \delta+\epsilon>0 \\
  \nonumber
   \frac{d(P_{e})}{d(\delta)} \quad at\quad \delta-\epsilon<0 \\
   \nonumber
   \therefore \delta=\frac{1}{2A}log\left(\frac{p}{1-p}\right)\longrightarrow   minima\\
   A=5 \implies \delta=\frac{1}{10}log\left(\frac{p}{1-p}\right)
 \end{align}
 \\
\item Repeat the above exercise using the MAP criterion. \\
\solution
\begin{align}
&P_{X|Y}\brak{x|y} \big|_{X=1}= \frac{P(Y=y|X=1)P(X=1)}{P(Y=y)}\\
\begin{split}P(Y=y)=P(Y=y|X=1)P(X=1)\\+P(Y=y|X=-1)P(X=-1)\end{split}\\
&P(Y=y|X=1)P(X=1)=p P(Y=A+N)\\
&=p\brak{\frac{1}{\sqrt{2 \pi}} e^{\frac{-(y-A)^2}{2}}}\\
&\therefore P_{X|Y}(x|y) \big|_{X=1}=\frac{p\brak{\frac{1}{\sqrt{2 \pi}} e^{\frac{-(y-A)^2}{2}}}}{P(Y=y)}\\
&P_{X|Y}(x|y) \big|_{X=-1}= \frac{P(Y=y|X=-1)P(X=-1)}{P(Y=y)}\\
&P(Y=y|X=-1)P(X=-1)=(1-p) P(Y=-A+N)\nonumber\\
&=(1-p)\brak{\frac{1}{\sqrt{2 \pi}} e^{\frac{-(y+A)^2}{2}}}\\
&\therefore P_{X|Y}(x|y) \big|_{X=-1}=\frac{(1-p)\brak{\frac{1}{\sqrt{2\pi}} e^{\frac{-(y+A)^2}{2}}}}{P(Y=y)}
\end{align}
Now comparing $a=P_{X|Y}(x|y) \big|_{X=-1}$ and $b=P_{X|Y}(x|y) \big|_{X=1}$, if $a>b, X=-1$ is more likely ,$a<b, X=1$ is more likely.\\
$p e^{\frac{-(y-A)^2}{2}}\gtrless(1-p)e^{\frac{-(y+A)^2}{2}}$\\
$\implies e^{2Ay}\gtrless\frac{1-p}{p}$\\
$\implies y\gtrless\frac{1}{2A}log\brak{\frac{1-p}{p}}$\\
$\delta=\frac{1}{2A}log\brak{\frac{1-p}{p}}$\\
$y>\delta \implies$ X=1 is more likely\\
$y<\delta \implies$ X=-1 is more likely
		\end{enumerate}
		\end{document}